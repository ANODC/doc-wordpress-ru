\chapter{Форум bbPress}
\label{sec:chapter_forum}

\textbf{Доски объявлений и форумы} – надежные инструменты, позволяющие создать онлайн-сообщества.

\section{Создание форума}
\label{sec:part_create_forum}

После установки bbPress добавит меню Форумы, Темы и Ответы в панель администратора WordPress.

Щелкаем по Форумы – Новый форум. Вы увидите экран, напоминающий область редактирования записей. Введите заголовок для вашего форума, а также его описание. После чего щелкните по кнопке Опубликовать.



\section{Вывод bbPress форума в фронтэнде WordPress}
\label{sec:part_forum_page}


Теперь, когда у вас есть несколько форумов, вы можете вывести на экран эти форумы во фронтэнде вашего сайта. Создайте новую страницу WordPress. Назовите ее как-нибудь, к примеру, Форум, Поддержка, Сообщество и т.д. Вставьте следующий шорткод на страницу:

\small
\begin{lstlisting}[frame=single, breaklines=true]
[ bbp-forum-index ]
\end{lstlisting}
\normalsize


Отключите комментарии и обратные ссылки для страницы, после чего опубликуйте ее.

Перейдите в раздел Внешний вид – Меню и добавьте эту страницу в навигационное меню.

Теперь, когда пользователь перейдет на эту страницу, он увидит полноценный форум.

\section{Пользовательские роли в bbPress}
\label{sec:part_forum_users}

bbPress идет вместе с предустановленными ролями, каждая со своими возможностями.

\begin{itemize}
\item Keymaster. Владелец сайта или администратор WordPress автоматически связывается с ролью Keymaster при установке bbPress. Keymaster может удалять и создавать форумы, создавать, редактировать, удалять все записи, темы, форумы.
\item Модераторы. Пользователи с ролью модератора имеют доступ к инструментам модерации, которые используются для управления форумами, темами и записями.
\item Участники. Стандартная пользовательская роль, участники могут создавать и редактировать свои собственные темы и ответы, могут добавлять темы в избранное и подписываться на темы.
\item Гости. Гости могут только читать форумы, темы и записи.
\item Заблокированные. Когда пользователь блокируется, все его возможности тоже блокируются. Такие пользователи могут читать публично доступные темы и ответы, но они не могут участвовать в обсуждении форума.
\end{itemize}