\Abbreviations

В настоящем документе применяют следующие термины с соответствующими определениями:
\begin{description}
\item [API] (программный интерфейс приложения, интерфейс прикладного программирования) (англ. application programming interface, API) — набор готовых классов, процедур, функций, структур и констант, предоставляемых приложением (библиотекой, сервисом) или операционной системой для использования во внешних программных продуктах.
\item [JSON] (англ. JavaScript Object Notation) текстовый формат обмена данными.
\item [GUI] (англ. Graphical User Interface) графический пользовательский интерфейс.
\item [ТТ] (англ. Trouble Ticket) Зарегистрированное обращение о проблеме или инциденте.
\item [АС] автоматизированные системы.
\item [АРМ] автоматизированное рабочее место.
\item [БД] база данных.
\item [ВМ] виртуальная машина.
\item [ЗИП] Запасные части, инструменты, принадлежности
\item [ИР] информационные ресурсы.
\item [ИТ] информационные технологии.
\item [НСД] несанкционированный доступ.
\item [ПК] Персональный компьютер
\item [ПО] программное обеспечение.
\item [СЗИ] средства защиты информации.
\item [СУБД] Система управления базами данных.
\item [СХД] система хранения данных.
\item [Узел сети] это объект сети (физический, виртуальный), который необходимо наблюдать. Это может быть физический сервер, сетевой коммутатор, виртуальная машина или какое-нибудь приложение.

\end{description}