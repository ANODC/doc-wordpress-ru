\chapter{Приложение}
\label{appendix:appendix}

\section{Плагины WordPress}
\label{sec:wp_plugins}

Для расширения функций WordPress используются плагины представленные ниже.
 
\subsection{Плагин Display Posts}
\label{sec:plugin_displayposts}

Плагин для отображения списка записей по категориям.

Адрес сайта: \url{https://displayposts.com/}

Руководство пользователя: \url{https://displayposts.com/docs/parameters/}

Пример настройки:
\small
\begin{lstlisting}[frame=single, breaklines=true]
[display-posts category_id="103" include_title="true" include_link="true" include_content="false" content_class="entry-content" order="ASC" include_excerpt_dash="true" excerpt_more="..." orderby="title" posts_per_page="1000"] 
\end{lstlisting}
\normalsize


\subsection{Плагин HTTP / HTTPS Remover}
\label{sec:plugin_http_https_remover}

Этот плагин создает относительные ссылки путем удаления http + https из URL.

Адрес сайта: \url{https://ru.wordpress.org/plugins/http-https-remover/}

\subsection{Плагин WP Statistics}
\label{sec:plugin_wp_statistics}

Полная аналитика и статистика для сайта на WordPress.

Адрес сайта: \url{https://ru.wordpress.org/plugins/wp-statistics/}

\subsection{Плагин PDF Embedder}
\label{sec:plugin_pdf_embedder}

Встраивайте PDF-файлы прямо в ваши записи и страницы, с гибкой шириной и высотой.

Адрес сайта: \url{https://ru.wordpress.org/plugins/pdf-embedder/}
Руководство пользователя: \url{https://wp-pdf.com/free-instructions/}

Пример настройки:
\small
\begin{lstlisting}[frame=single, breaklines=true]
[pdf-embedder url="https://mysite.com/wp-content/uploads/2015/01/Plan-Summary.pdf" width="500"]
\end{lstlisting}
\normalsize

\subsection{Плагин WP Media Category Management}
\label{sec:plugin_wp_media_category_management}

Плагин для обеспечения возможностей управления категориями медиафайлов в административном интерфейсе WordPress.

Адрес сайта: \url{http://ru.wordpress.org/plugins/wp-media-category-management/}

\subsection{Плагин Hide Featured Image}
\label{sec:plugin_hide_featured_image}

Плагин для отображения/скрытия избранных изображений на отдельных записях. Например избранные изображения отображаются, когда записи отображаются списком, но при входе в отдельную запись, избранное изображение не показывается.

Адрес сайта: \url{http://ru.wordpress.org/plugins/hide-featured-image/}

\subsection{Плагин Events Manager}
\label{sec:plugin_events_manager}

Плагин для регистрации событий и управление бронированием для WordPress. Позволяет создавать периодические события, указывать местоположения, использовать карты Google, RSS, поддерживает регистрацию бронирования и многое другое.

Адрес сайта: \url{http://ru.wordpress.org/plugins/events-manager/}

\subsection{Плагин Cerber Security, Antispam and Malware Scan}
\label{sec:plugin_serber}

Защищает WordPress от хакерских атак, спама, троянов и вредоносного ПО.
Предотвращает атаки перебора ограничением попыток входа через форму логина, XML-RPC/REST API и с использованием куки.
Отслеживает активность как пользователей так и нарушителей с уведомлениями по email, а также в браузере и на мобильных устройствах.
Защита от спама как собственными средствами, так и через. Google reCAPTCHA, для форм регистрации, форм контактов и комментариев.
Сканер на вредоносное ПО, проверка целостности и наблюдение за изменением файлов.
Усиление защиты WordPress гибкими правилами защиты и умными алгоритмами.
Ограничение доступа посредством черных и белых списков IP адресов.

Адрес сайта: \url{http://ru.wordpress.org/plugins/wp-cerber/}

\subsection{Плагин bbPress}
\label{sec:plugin_bbPress}

bbPress — простое и бесконечно мощное программное обеспечение для форума, созданное участниками разработки WordPress.
Форум bbPress прост в интеграции, лёгок в использовании и создан для масштабирования вместе с растущим сообществом.

Адрес сайта: \url{http://ru.wordpress.org/plugins/bbpress/}

\subsection{Плагин WP Super Cache}
\label{sec:plugin_wp_super_cache}

WP Super Cache самый популярный плагин для кеширования страниц в WordPress.
Плагин создаёт статичные html и php файлы – копии страниц WordPress и сохраняет их в кеш: /wp-content/cache/supercache/. Потом, при заходе пользователя на какую-либо страницу сайта, WordPress, вместо того, чтобы создать страницу с нуля, отдаёт браузеру заранее сохранённую копию html-страницы из кеша или собирает её максимально быстро из готовых php-файлов.

Адрес сайта: \url{http://ru.wordpress.org/plugins/wp-super-cache/}

\subsection{Плагин Yandex Mail}
\label{sec:plugin_yandex_mail}

Отправка писем через Yandex SMTP.

Адрес сайта: \url{http://ru.wordpress.org/plugins/yandex-mail/}

\subsection{Плагин Embed Plus for YouTube}
\label{sec:plugin_youtube}

Просмотр видео Youtube с сайта WordPress.

Адрес сайта: \url{http://ru.wordpress.org/plugins/youtube-embed-plus/}

\subsection{Плагин NextGEN Галерея}
\label{sec:plugin_NextGEN}

NextGEN Gallery — самый популярный плагин фотогалереи для WordPress, и, учитывая более 6 миллионов загрузок, это один из самых популярных WordPress плагинов в принципе. NextGEN позволяет вам создавать красивые галереи, у него много возможностей: загрузка больших изображений, группировка галерей в альбомы и многое другое.

Адрес сайта: \url{http://ru.wordpress.org/plugins/nextgen-gallery/}